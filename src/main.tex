\documentclass{article}

\usepackage[letterpaper,top=2cm,bottom=2cm,left=3cm,right=3cm,marginparwidth=1.75cm]{geometry}
\usepackage[bulgarian]{babel}
\usepackage[utf8]{inputenc}
\usepackage{amsmath}
\usepackage{amsthm}
\usepackage{amsfonts}
\usepackage{mdframed}
\usepackage[colorlinks=true, allcolors=blue]{hyperref}
\usepackage[nottoc]{tocbibind}
\usepackage{titlesec}
\usepackage{xcolor}

\usepackage{graphicx}
\graphicspath{{images/}}

\usepackage[style=alphabetic,backend=biber]{biblatex}
\addbibresource{literature.bib}

\definecolor{superlightred}{HTML}{F5F5F5}
\definecolor{superlightblue}{HTML}{F5F5F5}
\renewcommand\qedsymbol{$\mathit{QED}$}
\newmdtheoremenv[backgroundcolor=superlightblue,innertopmargin=0pt]
	{theorem}{Теорема}[section]
\newmdtheoremenv[backgroundcolor=superlightred,innertopmargin=0pt]
	{definition}{Определение}[section]
\newtheorem*{fact}{Факт}

\titleformat{\section}{\normalfont\bfseries\large}{\thesection.}{1em}{}

\title{\textbf{Заглавие}\\ \Large{Подзаглавие}}
\author{Автор}

\begin{document}

\maketitle
\tableofcontents
\newpage

\section{Примерно определение}

\begin{definition}
Нека R е пръстен и I е непразно подмножество на R, $I \subseteq R$. Ще казваме,
че I е ляв (десен) идеал на R, ако са изпълнени следните условия:
\begin{itemize}
	\item $\forall a, b \in I \implies a - b \in I$;
	\item $\forall a \in I, r \in R \implies ra \in I (ar \in I)$.
\end{itemize}
\end{definition}

\section{Примерна теорема}

\begin{theorem}[Даламбер] \label{dalamber}
	Полето на комплексните числа $\mathbb{C}$ е алгебрически затворено.
\end{theorem}

\begin{proof}
	Доказателството е тривиално - предоставено е като упражнение за читателя.
\end{proof}

\section{Примерен факт}

\begin{fact}
	Математиката е много яко нещо!
\end{fact}

\par Случайно цитиране - \autocite{reed1960}.

\begin{figure}
\begin{center}
\includegraphics[height=100px]{test-image.jpg}
\caption{Картинка}
\label{fig:test-image}
\end{center}
\end{figure}

\par Там \ref{fig:test-image} има картинка!

\section{Литература}
\nocite{*}
\printbibliography[heading=none]

\end{document}
